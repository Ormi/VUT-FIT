%%%%%%%%%%%%%%%%%%%%%%%%%%%%%%%%%%%%%%%
% Project N°1 for IFY
% Author: Michal Ormos
% Date : 25.2.2015
%%%%%%%%%%%%%%%%%%%%%%%%%%%%%%%%%%%%%%%

\documentclass[a4paper, 11pt, twocolumn]{article}
\usepackage[czech]{babel}
\usepackage[utf8]{inputenc}
\usepackage[left=2cm,text={17cm,24cm},top=2.5cm]{geometry}
\usepackage[IL2]{fontenc} 
\usepackage{times}
\providecommand{\uv}[1]{\quotedblbase #1\textquotedblleft}


%\section{Font Styles}
\title{Typografie a publikování\\1. projekt} 
\author{Michal Ormoš\\xormos00@stud.fit.vutbr.cz} 
\date{}

\begin{document}
\maketitle

\section{Hladká sazba}
Hladká sazba je sazba z jednoho stupně, druhu a řezu písma sázena na stanovenou šířku odstavce. Skláda se z odstavců, které obvykle začínají zarážkou, ale mohou být sázeny i bez zarážky -- rozhodujíci je celková grafická úprava. Odstavce jsou ukončeny východovou řádkou. Věty nesmějí začínat číslicí.

Barevné zvýraznění, podtrhávání slov či různe velikosti písma vybraných slov se zde také nepoužívá. Hladká sazba je určena především pro delší texty, jako je naprříklad beletrie. Porušení konzistence sazby působí v textu rušivě a unavuje čtenárův zrak.

\section{Smíšená sazba}
Smíšná sazba má o něco volnější pravidla, jak hladká sazba. Nejčastěji se klasická hladká sazba doplňuje o další řezy písma pro zvýraznění důležitých pojmů. Existuje \uv{pravidlo}:
\begin{quotation}
\textsc{Čím více druhů, řezů, velikostí, barev písma a jiných efektů použijeme, tím profesionálněji bude dokument vypadat. Čtenář tím bude vždy nadšen!}
\end{quotation}

Tímto pravidlem se \underline{nikdy} nesmíte řídit. Píliš časté zvýrazňování textových elemntů a změny {\Huge V}{\huge E}{\Large L}{\large I}{\normalsize K}{\small O}{\footnotesize S}{\scriptsize T}{\tiny I} {\normalsize písma} {\Large jsou} {\huge známkou} {\huge \textbf {amatérismu}} autora a působí \textbf{\emph{velmi}} \emph{rušivě}. 
Dobře navržený dokument nemá obsahovat více než 4 řezy či druhy písma. \texttt{Dobrě navržený dokument je decentní, ne chaotický.}

Důležitým znakem správně vysázeného dokumentu je konzistentní pužívaní různych druhů zvýraznění. To naprříklad může znamenat, že \textbf{tučný řez} písma bude vyhrazen pouze pro klíčová slova, \emph{skloněný řez} pouze pro doposud neznámé pojmy a nebude se to míchat.
Skloněný řez nepůsobí tak rušivě a používá se častěji. V \LaTeX u je sázíme reději přikazem \verb|\emph{text}| než \verb|\textit{text}.|

Smíšená sazba se nejčastěji používá pro sazbu vědeckých článků a technických zpráv. U delších dokumentů vědeckého či technického charakteru je zvykem upozornit čtenáře na význam různych typů zvýraznění v údovní kapitole.

\section{České odlišnosti}
Česká sazba se oproti okolnímu světu v některých aspektech mírně liší. Jednou z odlišností je sazba uvozovek. Uvozovky se v češtině používají převážne pro zobrazení příme reči. V menší míře se používají také pro zvýraznění přezdívek a ironie. V češtině se používá tento \uv{typ uvozovek} namísto anglických "uvozovek".

Ve smíšené sazbě se řez uvozovek řídí řezem prvního uvozovaného slova. Pokud je uvozována celá věta, sází se koncová tečka před uvozovkou, pokud se uvozuje slovo nebo část věty, sází se tečka za uvozovkou.

Druhou odlišností je pravidlo pro sázení konců řádku. V české sazbě by řádek neměl končit osamocenou jednopísmennou předložkou nebo spojkou (spojkou \uv{a} končit může při sazbě do 25 liter). Abychom \LaTeX u zabránili v sázení osamocených předložek, vkládáme mezi předložku a slovo nezlomitelnou mezeru znakem \verb|~|\ (vlnka, tilda). Pro automatické doplnění vlnek slouží volně šiřitelný program \emph{vlna} od pana Olšáka.\footnote{Viz ftp://math.feld.cvut.cz/pub/olsak/vlna/.}

\end{document}
