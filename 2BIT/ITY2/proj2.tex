%%%%%%%%%%%%%%%%%%%%%%%%%%%%%%%%%%%%%%%
% Project N°2 for ITY
% Author: Michal Ormos
% Date : 30.10.2015
%%%%%%%%%%%%%%%%%%%%%%%%%%%%%%%%%%%%%%%

\documentclass[a4paper, 11pt]{article}
\usepackage[left=1.5cm,text={18cm,25cm},top=2.5cm]{geometry}
\usepackage[czech]{babel}
\usepackage[utf8]{inputenc}
\usepackage[IL2]{fontenc} 
\usepackage{times}
\usepackage{amsmath}
\usepackage{amsthm}
\usepackage{amsfonts}
\theoremstyle{definition} %link with 3 newtheorem
\newtheorem{veta}{Věta}   %making bold definitions, algoritm, sentence
\newtheorem{definice}{Definice}[section] %also numbers
\newtheorem{algoritmus}[definice]{Algoritmus}

%%%%
%Opravit cislovanie stran
%
%%%%%
\begin{document}

\onecolumn
\pagestyle{empty}
\begin{center}
\huge
\textsc{Fakulta informačních technologií
\\Vysoké učení technické v~Brně}

\vspace{\stretch{0.382}}
\LARGE
Typografie a publikování -- 2. projekt
\\Sazba dokumentů s~matematickými výrazy
\vspace{\stretch{0.618}}
\end{center}

{\LARGE 2015 \hfill Michal Ormoš}


\newpage
\pagestyle{plain}
\setcounter{page}{1}

\twocolumn

\section*{Úvod}
V~této úloze si vyzkoušíme sazbu titulní strany, matematických vzorců, prostředí a dalších textových struktur obvyklých pro technicky zaměřené texty (například rovnice (\ref{rovnice1}) nebo definice \ref{definice1.1} na straně \pageref{definice1.1}).

Na titulní straně je využito sázení nadpisu podle optického středu s~využitím zlatého řezu. Tento postup byl probírán na přednášce.


\section{Matematický text}
Nejprve se podíváme na sázení matematických symbolů a výrazů v~plynulém textu. Pro množinu $V$ označuje card($V$) kardinalitu $V$.
Pro množinu $V$ reprezentuje $V^{*}$ volný monoid generovaný množinou $V$ s~operací konkatenace.
Prvek identity ve volném monoidu $V^{*}$ značíme symbolem $\varepsilon$.
Nechť $V^{+} = V^{-} - \{\varepsilon\}$. Algebraicky je tedy $V^{+}$ volná pologrupa generovaná množinou $V$ s~operací konkatenace.
Konečnou neprázdnou množinu $V$ nazvěme \textit{abeceda}.
Pro $w \in V^*$ označuje $|w|$ délku řetězce  $w$. Pro $W \subseteq V$ označuje $\mbox{occur}(w,W)$ počet výskytů symbolů z ~$W$ v~řetězci $w$ a $\mbox{sym}({w,i})$ určuje $i$-tý symbol řetězce $w$; například $\mbox{sym}(abcd,3)=c$.

Nyní zkusíme sazbu definic a vět s~využitím balíku \texttt{amsthm}.

\begin{definice}
\label{definice1.1}
\emph{Bezkontextová gramatika} je čtveřice $G=(V,T,P,S)$, kde $V$ je totální abeceda, $T \in V$ je abeceda terminálů, $S \subseteq (V-T)$ je startující symbol a $P$ je konečná množina pravidel tvaru $q\colon A~\rightarrow \alpha$, kde $A \in (V~- T) $, $\alpha \in V^*$ a $q$ je návěští tohoto pravidla. Nechť $N = V~- T$ značí abecedu neterminálů.
Pokud $q\colon A~\rightarrow \alpha \in P$, $\gamma$ , $\delta \in V^*$ provádí derivační krok z~$\gamma A~\delta$ do $\gamma \alpha \delta$ podle pravidla $q\colon A~\rightarrow \alpha$, symbolicky píšeme 
$\gamma A~\delta \rightarrow \gamma \alpha \delta \ [q\colon A~\rightarrow \alpha]$ nebo zjednodušeně $\gamma A~\delta \Rightarrow \gamma \alpha \delta$. Standardním způsobem definujeme $\Rightarrow^m$, kde $m \geq 0 $ . Dále definujeme 
tranzitivní uzávěr $\Rightarrow^+$ a tranzitivně-reflexivní uzávěr $\Rightarrow^*$.
\end{definice}

Algoritmus můžeme uvádět podobně jako definice textově, nebo využít pseudokódu vysázeného ve vhodném prostředí (například \texttt{algorithm2e}).

\begin{algoritmus}
Algoritmus pro ověření bezkontextovosti gramatiky. Mějme gramatiku $G = (N, T, P, S)$.
\begin{enumerate}
\item \label{krok1} Pro každé pravidlo $p \in P$ proveď test, zda $p$ na levé straně obsahuje právě jeden symbol z~$N$ .
\item Pokud všechna pravidla splňují podmínku z~kroku \ref{krok1}, tak je gramatika $G$ bezkontextová.
\end{enumerate}
\end{algoritmus}

\begin{definice}
Jazyk definovaný gramatikou $G$ definujeme jako $L(G) = \{w \in T^*\ |\ S~\Rightarrow^* w \} $ .
\end{definice}

\subsection{Podsekce obsahující větu}

\begin{definice}
Nechť $L$ je libovolný jazyk. $L$ je \textit{bezkontextový jazyk}, když a jen když $L = L(G)$, kde $G$ je libovolná bezkontextová gramatika.
\end{definice}

\begin{definice}
Množinu $\mathcal{L}_{CF} = \{L | L$  je bezkontextový
jazyk\} nazýváme \textit{třídou bezkontextových jazyků}.
\end{definice}

\begin{veta}
Nechť $L_{abc} = \{a^n b^n c^n \ | n \geq 0 \}$ Platí, že $L_{abc} \not\in \mathcal{L}_{CF}$. \label{veta1}
\end{veta}

\begin{proof}
Důkaz se provede pomocí Pumping lemma pro bezkontextové jazyky, kdy ukážeme, že není možné, aby platilo, což bude implikovat pravdivost věty \ref{veta1}.
\end{proof}

\section{Rovnice a odkazy}

Složitější matematické formulace sázíme mimo plynulý text. Lze umístit několik výrazů na jeden řádek, ale pak je třeba tyto vhodně oddělit, například příkazem \verb|\quad|. 

$$\sqrt[x^2]{y^3_0} \quad \mathbb{N} = \{1,2,3,\ldots \} \quad x^{y^y} \neq x^{yy} \quad z_{i_j} \not\equiv {z_i}_j$$

V~rovnici (\ref{rovnice1}) jsou využity tři typy závorek s~různou explicitně definovanou velikostí.

\begin{eqnarray}
\bigg\{\Big[\big(a + b\big) * c\Big]^d + 1\bigg\}& = & x\label{rovnice1}
\end{eqnarray}
\begin{eqnarray*}
\lim_{x \to \infty}\frac{\sin^2 x + \cos^2 x}{4} &=& y
\end{eqnarray*}

V~této větě vidíme, jak vypadá implicitní vysázení limity $\lim_{n \to \infty}f(n)$ v~normálním odstavci textu. Podobně je to i s~dalšími symboly jako $\sum_1^n$ či $\bigcup_{A\in \mathcal{B}}$ . V~případě vzorce $\lim\limits_{x \to 0}\frac{\sin x}{x}=1$ jsme si vynutili méně úspornou sazbu příkazem \verb|\limits|.

\begin{eqnarray}
 \int\limits_a^b f(x)\,\mathrm{d}x &=& -\int_b^a f(x)\,\mathrm{d}x  \\
\left(\sqrt[5]{x^4}\right)' = \left(x^{\frac{4}{5}}\right)' &=& \frac{4}{5}x^{-\frac{1}{5}} = \frac{4}{5\sqrt[5]{x}}\\
\overline{\overline{A \vee B}} &=& \overline{\overline{A} \wedge \overline{B}}
\end{eqnarray}

\section{Matice}

Pro sázení matic se velmi èasto používá prostøedí \texttt{array} a závorky  (\verb|\left|, \verb|\right|). 

 $$\left( \begin{array}{cc}
a + b & a - b \\
\widehat{\xi + \omega} & \hat{\pi} \\
\vec{a} & \overrightarrow{AC} \\
0 & \beta\\
\end{array} \right)$$

$$\mathbf{A}=\left\|\begin{array}{cccc}
a_{11} & a_{12} & \ldots & a_{1n} \\
a_{21} & a_{22} & \ldots & a_{2n} \\
\vdots & \vdots & \ddots & \vdots \\
a_{m1} & a_{m2} & \ldots & a_{mn}
\end{array}\right\|$$

$$\left|\begin{array}{cc}
t & u~\\ v~& w
\end{array}\right| = tw -uv$$


Prostředí \texttt{array} lze úspěšně využít i jinde.

$$\binom{n}{k} = \left\{ 
\begin{array}{l l}
  \frac{n!}{k!(n-k)!} & \quad \mbox{pro $0 \leq k~\leq n$}\\
  0 & \quad \mbox{pro $k < 0$ nebo $k > n$}\\
\end{array} \right. $$

\section{Závěrem}

V~případě, že budete potřebovat vyjádřit matematickou konstrukci nebo symbol a nebude se Vám dařit jej nalézt v~samotném \LaTeX u, doporučuji prostudovat možnosti balíku maker \AmS-\LaTeX. 
Analogická poučka platí obecně pro jakoukoli konstrukci v~\TeX u.

\end{document}
