%%%%%%%%%%%%%%%%%%%%%%%%%%%%%%%%%%%%%%%
% Project N°4 for ITY
% Author: Michal Ormos
% Date : 8.4.2015
%%%%%%%%%%%%%%%%%%%%%%%%%%%%%%%%%%%%%%%

\documentclass[a4paper, 11pt]{article}
\usepackage[left=2cm,text={17cm,24cm},top=3cm]{geometry}
\usepackage[czech]{babel}
\usepackage[T1]{fontenc}
\usepackage[utf8]{inputenc}
\usepackage{times}
\bibliographystyle{czplain}
\providecommand{\uv}[1]{\quotedblbase #1\textquotedblleft}

\begin{document}

%%%% Title Page %%%%
\begin{titlepage}
\begin{center}
\huge
\textsc{\Huge Vysoké učení technické v~Brně
\\\huge Fakulta informačních technologií}

\vspace{\stretch{0.382}}
\Large{Typografia a publikovanie -- 4. projekt}\\
\Huge{Bibliografické citácie}
\vspace{\stretch{0.618}}
\end{center}

{\Large \today \hfill Michal Ormoš}
\end{titlepage}
%%%% Title Page %%%%

\newpage

\section{Úvod do teórie písma}
Písmo vzniklo z prirodzenej potreby trvalejšieho zaznamenávania ľudskej reči a ďalšieho sprostredkovania myšlienok a udalostí v ich živote \cite{Wiki:pismo}. Preklad slova \uv{písmo} z anglického jazyka \uv{font} (ktorý tu budem často zmieňovať). V našej geografickej oblasti vplyvom rímskej ríše sa ukotvila hlavne latinka \cite{primo:Muzika}.

Počas rôznych historických období a spoločenských či technický zmien bola zmena a vývoj písma neustále potrebná \cite{primo:Jean}. V dnešnej počítačovej spoločnosti sa už písmo rapidne nevyvíja kedže máme k dispozícii veľkú škálu rôznych fontov a typov písma. Ľudia si väčšinou zvýknú na pár pravidelne zaužívaných fontov a prehnane točené či klopené formy písma nie su až tak pre nich prirodzené a prívetivé.

\section{Propagácia písma v novoveku}
Zaujímavé, zábavné a poučné články v obore písma môžeme nájsť vo veľa publikáciách napríklad \cite{primo:article1} alebo zborník z konferencie \cite{primo:conf}.

Hlavný záujem využitia a vývoja písma je v dnešnej dobe v reklamných a propagačných formách \cite{primo:fonts}. Kedže vyzuálna stránka pri zaujatí davu tvorí veľkú časť úspechu. Rôzne typy, veľkosti či farby. Trochu nezvyčajné sklony alebo známe a často používané fonty môžu cieľovú skupinu ľahko zaujať \cite{Wiki:typeface} a teda aj priniesť úspech propagovanému predmetu. 

Další smer využitia fontu je v modernej digitálnej typografii. Od raziacich hláv v minulosti cez písacie stroje až po pohodlnú klávesnicu v dnešných časoch. Digitálne spracovanie písma je veľmi jednoduché pomocou moderných grafikcých aplikácií. Testovanie a použitie ešta viac.

\section{Štúdia písma}
Štúdiou písma sa zaoberá veľa inštitúcii a taktiež tvorí tému bakalárksych či diplomových prác \cite{thesis:Jiricek} \cite{thesis:Zankofski}. Tvorí zaujímavú oblasť štúdia a vývoja. Nato aby sa človek mohol v takomto obore vyznať mal by si určite naštudovať základny slovník pojmov aby sa vedel orientovať \cite{misc:dictionary}.

Stále je tu priestor na zdokonaľovanie, zkrášlovanie a vývoj písma vo všetkých smeroch aj napriek nie tak rapidnému rozvoju ako bol v minulosti. Charakteristiku písma tvorí hlavne jeho sklon, šírka, optická veľkosť, pätky a metriky.

\newpage
\bibliography{literatura}

\end{document}
